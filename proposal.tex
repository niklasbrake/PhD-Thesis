\section*{From Thesis Proposal}
\red{Below is from my thesis proposal for my comprehensive exam. It needs to be reworked for the thesis.}

The immune system, a disperse, complex network of interacting cellular and humoral factors, hosts a variety of interesting dynamic properties with the goal of protecting the host from harmful pathogens and cancerous cells, all the while remaining tolerant toward the rest of the body (i.e. toward ``self''). Abnormal behaviour of one or more of these elements can lead to undesirable outcomes, ranging from failure to clear malignant pathogens (leading to chronic infections or death of the host) at one extreme, to the development of autoimmune disorders at the other. Understanding how all these elements work together to generate the immune response is imperative for the development of therapeutics targeting a wide variety of infections and/or disorders.

The development and analysis of mathematical and computational models, simple and detailed alike, have proved useful in understanding such immunological systems. In the past, these models have allowed us to build upon existing knowledge, and answer questions that cannot be easily addressed otherwise. Since traditional experimental approaches in the field of immunology often find their limits in the available methodology, quantitative approaches can sometimes fill in the gaps, test various hypotheses, provide possible explanations for observed phenomena, and offer predictions for how the system might be behaving.

The goal of this Ph.D. thesis, which will be outlined in detail in this project proposal, is to explore this computational avenue by considering how different model types and methodologies can, from an immune population dynamics standpoint, provide an alternate perspective to three unique problems. The first such problem involves the construction of a compartmentalized ordinary differential equation (ODE) model to analyze the kinetics of immunoregulation during simultaneous autoimmune dysfunctions; the second presents an integro-differential equation (IDE) model to study a continuum of T-cell avidity distributions and their role in the progression and resolution of acute or chronic viral infections; and [TCR diversity project]

%%%%%%%%%%%%%%%%%%%%%%%%%%%%%%%%%%%%%%%%%%%%%%%%%%

The integration of mathematics and biology is a discipline that is becoming increasingly relevant, with mathematical techniques being applied to a vast range of physiological systems at different levels: from the genetic and molecular to the tissue and systems levels (for a broad overview, see~\cite{keener1998mathematical}). Particularly, within the field of immunology, significant progress has been made in deciphering immune cell responses via quantitative modelling. In the following sections, I will provide an overview of the ways in which mathematical models have been applied to study immune cell population dynamics. I will also present some physiological background and ongoing problems in the field that we intend to pursue using the tools outlined herein.

\subsection*{Population modelling in the context of autoimmune disorders}

An autoimmune disorder results from the failure of the immune system to remain tolerant to cells belonging to the host. Well-known examples of autoimmune disease include, but are not limited to, type 1 diabetes mellitus, multiple sclerosis, rheumatoid arthritis, Hashimoto's thyroiditis, autoimmune hepatitis, and inflammatory bowel disease. In short, these disorders arise when cells of the immune system pathologically induce the destruction of tissues that serve vital physiological functions, leading to a multitude of adverse outcomes.

Commonly (though not exclusively), mathematical models in the context of autoimmune disorders such as multiple sclerosis~\cite{elettreby2020simple,pernice2019computational}, type 1 diabetes~\cite{khadra2009role,khadra2011investigating,jaberi2014autoimmune}, rheumatoid arthritis~\cite{baker2013mathematical} or even generalized immune and autoimmune models~\cite{alexander2011self,lorenzi2015mathematical,kim2007modeling} are written as ordinary differential equation models. These models typically describe the evolution of immune cell populations, target cell populations, and/or cytokine/chemokine concentrations in time, and present the various model outcomes in different parameter regimes. While such models often leave out many elements for simplicity, they remain advantageous as they have shorter simulation times, are easier to investigate analytically and numerically, and still have important predictive abilities.

T cells are key players in the development, or in the prevention, of autoimmune diseases. Importantly, the immune system comprises not only effector T cells that induce destruction of the target tissue, but also regulatory T cells (such as the CD25\pos, Foxp3\pos{} T cells) that promote tolerance by acting as suppressors of effector immune cells (reviewed in~\cite{li2015foxp3+}). Impaired or insufficient regulatory T cell function has been strongly linked with the development of autoimmune disorders~\cite{bennett2001immune, buckner2010mechanisms}, and the balance between effector and regulatory T cells is often considered in mathematical models of immunity and autoimmunity~\cite{su2009mathematical,jaberi2015continuum,pernice2019computational,khadra2009role,alexander2011self, kim2007modeling}.

\subsection*{Immunoregulation in simultaneously induced autoimmune disorders}

In many cases, the search for an effective treatment against autoimmunity targets regulatory T cells in some way (see~\cite{arellano2016regulatory} for review). Currently, however, common treatments for autoimmune diseases, such as the use of immunosuppressant drugs, generally have the disadvantage of suppressing the body’s entire immune system~\cite{wraith2016antigen,Tr1paper}, leaving the patient more susceptible to infections. %Furthermore, in autoimmune disorders such as in type 1 diabetes or multiple sclerosis, a wide variety of cell types and tissue-specific autoantigens are implicated in the progression of these diseases~\cite{santamaria2010long,tsai2008cd8,babbe2000clonal} which complicates the implementation of effective immunotherapies targeting them.
Recent studies have reported that intravenous administration of pMHC class II (pMHCII)-coated nanoparticles (NPs) results in the expansion of T-regulatory type 1 (Tr1)-like cells, and that these cells are successful at suppressing autoimmune disorders in a disease-specific manner~\cite{Tr1paper,umeshappa2019suppression,umeshappa2020ubiquitous,singha2017peptide,serra2015nanoparticle}, without impairing global immunity~\cite{Tr1paper,bayry2016repressing}. Unlike their Foxp3\pos{} counterparts, Tr1 cells form a different type of regulatory CD4\pos{} T cell that are characterized by co-expression of CD49b and LAG-3 cell surface proteins both in humans and in mice (among other markers)~\cite{gagliani2013coexpression} and have been known to exhibit regulatory function since their first description in 1994 by Bacchetta et al.~\cite{bacchetta1994high}.

% Our recent studies in mice treated with autoimmune disease-relevant peptide-major histocompatibility complex class II (pMHCII)-based nanomedicines have shown that autoantigen-specific Tr1 cells can blunt the progression of organ-specific autoimmunity without compromising systemic immunity and that they do so, at least in part, by suppressing the pro-inflammatory and antigen presenting capacity of cognate autoantigen-loaded professional antigen-presenting cells (APCs)~\cite{Tr1paper,bayry2016repressing}.

Systemic delivery of these pMHC-NPs re-programs endogenous autoantigen-experienced CD4$^+$ T cells into Tr1-like cells, followed by expansion of the resulting Tr1 cells, their recruitment into the autoimmune-targeted organ and associated lymphoid tissue, and comprehensive suppression of local inflammation~\cite{Tr1paper,serra2015nanoparticle,singha2017peptide,umeshappa2019suppression,umeshappa2020ubiquitous}. The mechanisms underlying the latter are complex but involve suppression of autoantigen-loaded APCs and local B-regulatory-cell formation by Tr1-cell-derived IL-10, TGF-\textbeta{} and IL-21 secreted in response to recognition of cognate pMHCII on local autoantigen-loaded APCs~\cite{Tr1paper}. %pMHCII-NP therapy can also reprogram human antigen-experienced CD4\pos{} T cells in NOD scid Il2rg$^{-/-}$ mice reconstituted with peripheral blood mononuclear cells (PBMCs) from patients with autoimmune type 1 diabetes~\cite{Tr1paper} or autoimmune liver disease~\cite{umeshappa2019suppression}, providing support for the translational potential of this therapeutic avenue.

In the case of pMHCII-NPs displaying epitopes from tissue specific autoantigens (i.e. only expressed in specific cells of the body), the expanded Tr1-like cells successfully inhibited the progression of autoimmunity in that tissue, but not in others~\cite{wraith2016antigen,Tr1paper,bayry2016repressing}. One would thus expect that pMHCII-NPs displaying systemically expressed autoantigenic epitopes, such as those targeted by PDC-E2-autoreactive T-cells in liver autoimmunity, would be able to suppress autoimmunity in various organs. In fact, pMHCII-NPs displaying epitopes from ubiquitously expressed autoantigens, including the mitochondrial PDC-E2 complex, can suppress not only a wide range of liver autoimmune diseases~\cite{umeshappa2019suppression}, including autoimmune hepatitis (AIH), primary biliary cholangitis (PBC), and primary sclerosing cholangitis, but also extra-hepatic autoimmunity~\cite{umeshappa2020ubiquitous}. PDC-E2 was described as a key autoantigen in the development of primary biliary cirrhosis~\cite{yeaman1996pyruvate}, though it is a ubiquitous self-antigen located in mitochondria. In both experimental autoimmune encephalomyelitis (EAE) or AIH models, Tr1 cells induced in response to PDC-E2-based pMHCII-NPs (referred to hereafter as PDC-NPs) accumulated into the target organ and/or proximal draining lymph nodes~\cite{umeshappa2019suppression,umeshappa2020ubiquitous}. In mice simultaneously having AIH and EAE, however, these compounds selectively blunted liver autoimmunity~\cite{umeshappa2020ubiquitous}.

Myelin oligodendrocyte glycoprotein (MOG) is a CNS-specific autoantigen that is used to induce EAE~\cite{mendel1995myelin}. Contrary to experiments using PDC-NPs, treatment of comorbid mice having both AIH and EAE with MOG-based pMHCII-NPs (referred to hereafter as MOG-NPs) triggered the formation and recruitment of cognate Tr1 cells to the CNS-draining lymph nodes, and reversed CNS autoimmunity without suppressing liver inflammation~\cite{umeshappa2020ubiquitous}. However, when these studies were repeated in comorbid mice having EAE and a more severe form of liver autoimmunity than AIH (namely, PBC), the MOG-specific Tr1 cells induced by MOG-NPs also failed to suppress the progression of EAE~\cite{umeshappa2020ubiquitous}. It is therefore desirable to explain these puzzling and wide-ranging outcomes by generating a compartmentalized population model representing this system.

% Our goal in this study is to explore how interactions between physiological variables of interest, namely, Tr1 cells, effector T cells, and APCs, and variations in physiological parameters, including tissue size, Tr1 cell recruitment, retention and accumulation, and local versus systemic autoantigen expression bring about such outcomes. Given the high occurrence of ``polyautoimmunity'' in the human population~\cite{rojas2012introducing}, we wish to understand how these pMHCII-NP-expanded Tr1 cells traffic to multiple sites of inflammation. To accomplish this, we develop a mathematical model composed of a system of nonlinear ordinary differential equations based on the interactions of several cell populations to study how they impact dynamics. The model is compartmentalized into separate cell pools, each representing the organs under consideration (namely, the liver, brain, and bloodstream). Using the results of the model, we evaluate the validity and significance of the above hypotheses and underline the interplay between Tr1-cell allocation and immunosuppressive efficacy.

\subsection*{T-cell modelling in the context of infectious diseases}

Similar to mathematical models used to investigate autoimmune responses, studies that model infectious disease dynamics often make use of ordinary differential equation modelling. A well-known model of viral infection dynamics, for example, which has since been coined the ``target cell-limited model,'' is a simple, three-variable ODE model originally intended to describe the kinetics of hepatitis B virus~\cite{nowak1996viral} and human immunodeficiency virus (HIV-1) infection~\cite{nowak1996population}. This
model has since been modified in several studies to fit viral data of different acute and chronic pathogens, including HIV-1, hepatitis C virus (HCV), H1N1, Ebola virus (EBOV), Dengue virus (DENV) and Zika virus (ZIKV)~(as reviewed in~\cite{zitzmann2018mathematical}), and recently even severe acute respiratory syndrome coronavirus (SARS-CoV-2)~\cite{wang2020modeling}, the novel virus responsible for the ongoing COVID-19 pandemic.

However, models in immunology, including those relevant to infectious disease dynamics as well as to autoimmune disorders, are often beyond the scope of standard ODE modelling. Examples of more complex mathematical tools in the literature include the use of partial differential equation~\cite{su2009mathematical,moise2019rheumatoid}, integro-differential equations~\cite{jaberi2015continuum,delitala2013mathematical}, and delay-differential equations~\cite{bocharov1998modelling,gourley2008dynamics}. While these tend to be more difficult to simulate and to analyze, they can also provide more information about the systems being studied (e.g. spatiotemporal considerations, heterogeneous antigen expression or affinity profiles, or non-instantaneous feedback). Of particular interest to us is the ability to track the evolution of a collection of antigen-specific effector T cells that together form a heterogeneous distribution of binding avidity with their cognate ligands. 

\subsection*{The role of T-cell receptor diversity in viral infections}

The generation of lymphocyte receptor diversity is a key feature of adaptive immunity (reviewed in~\cite{cooper2006evolution,schatz2011recombination}). In addition to somatic recombination, diversification of the T cell repertoire is significantly enhanced by a unique DNA polymerase called terminal deoxynucleotidyl transferase, or TdT~\cite{cabaniols2001most}. The genetic sequence and structure of TdT has been highly conserved in vertebrates~\cite{lee1994isolation,hansen1997characterization}, suggesting an evolutionary benefit for its ability to diversify the T cell repertoire through non-templated nucleotide additions. However, TdT-knockout studies have shown that lack of TdT does not abrogate T cell immunity in response to acute stimulation by bacterial or viral antigen~\cite{gilfillan1995efficient}, and so the specific advantage that this added TdT-dependent variability brings forth remains unclear.

The question that arises is then whether or not the lack of TdT would impair T cell responses to a chronic pathogen instead. To this effect, a good model organism is that of the lymphocytic choriomeningitis virus (LCMV), of which the two most famous strains, namely Armstrong (Arm) and clone 13 (Cl13),  produce an acute or chronic infection in mice, respectively~\cite{abdel2019viruses}. Many experimental breakthroughs in the field of immunology, with the important example of the discovery of MHC-restriction~\cite{zinkernagel1974restriction} for which the Nobel Prize was awarded to Peter Doherty and Rolf Zinkernagel in 1996, were accomplished through analysis of the immune response to LCMV~\cite{abdel2019viruses,bocharov2015understanding}. There have similarly been a handful of mathematical studies that specifically use LCMV kinetic data to study outcomes of the immune response against a viral pathogen, with the models outlined in~\cite{bocharov1998modelling,kecsmir2003clonal,de2003different,ludewig2004determining} serving as examples.

Preliminary experimental results obtained from knockout mice in the Mandl Lab (McGill University, unpublished) suggest that T cell diversification by TdT might generate a T cell receptor (TCR) repertoire with lower pMHC binding affinity on average. What remains to be seen is whether lower avidity T cells might play a more important role in providing immunity against chronic antigen stimulation. Furthermore, observations from other studies suggest that some TdT-dependent T cells may in fact bind pMHC with a higher affinity than their TdT-independent counterparts~\cite{conde1998terminal,feeney2001terminal}. In the proposed research, we plan to explore various hypotheses for TdT-induced differences in T cell avidity distributions, and to understand the relationship between T cell-pMHC binding avidity and the duration of antigen exposure (i.e. acute vs. chronic) with the hopes of uncovering the possible benefits of TdT diversification for T cell function.

% Many theoretical studies have previously modelled viral dynamics, few of which focused on infections by acute and/or chronic lymphocytic choriomeningitis virus (LCMV). We simulated the results of some of these models described in~\cite{nowak1995antigenic,zitzmann2018mathematical,ciupe2018bistable,bocharov1998modelling, kecsmir2003clonal} to form a framework upon which we can construct our avidity-centric model. However, these models were often either too complicated, focused on elements of less relevance to us, and/or lacked the bistability feature, and essential component that could allow low-dose LCMV clone 13 inoculations to generate no response, consistent with~\cite{stamm2012intermediate}.







%%%%%%%%%%%%%%%%%%%%%%%%%%%%%%%%%%%%%%%%%%%%%%%%%%

\subsection*{Key aims}

The goals of this thesis are ultimately to inform the underlying biology of relevant immunological system by means of mathematical techniques. In the previous section, various forms of mathematical models were discussed, and outstanding problems and questions in the field of immunology were raised. By combining the two disciplines, we hope to shed light on the inner workings of the immune system. This general aim encompasses three distinct goals, as outlined below.

\textbf{Goal 1:} Create an ODE model to rationalize the outcomes of pMHCII-NP-generated Tr1 immunoregulation in the context of simultanous hepatic and extra-hepatic autoimmunity. The model is compartmentalized into separate cell pools, each representing the organs under consideration (namely, the liver, brain, and bloodstream). In using such equations to describe physiological variables of interest (Tr1 cells, effector T cells, and APCs), we can study the role of key biological parameters (e.g. Tr1 cell recruitment, retention and accumulation, local versus systemic autoantigen expression, and Tr1 binding strength) in bringing about the non-trivial outcomes associated with pMHCII-NP therapy. 

\textbf{Goal 2:} Study the response not just of a single T cell clone against a viral pathogen, but of a population of T cells that together display an entire continuum of T-cell avidities. By constructing a system of integro-differential equations, we can not only consider the temporal evolution of key population sizes (namely effector T-cell levels and viral loads), but also track the evolution and infer the role of the distribution of T cell avidities over time. Such a model will allow us to investigate hypotheses regarding the benefits that TdT-dependent diversification of the T cell repertoire may confer, by testing populations of different starting distributions against acute or chronic viral challenge.

\section*{Mathematical models to weave into the introduction}

\begin{enumerate}

 \item Regulatory T cells/autoimmunity:
\begin{itemize}
    \item Leon et al., J Theor Biol 2000 and J Autoimmunity 2004, propose a model explaining that epidemiological instances of autoimmunity can be explained by regulatory and effector T cell coexistence of ``tolerant'' and ``autoimmune'' states; which one would depend on factors such as the level of self-antigen expression and the preferential induction of E vs. R proliferation  - the role of bistability will become important also in our model.
    \item 
\end{itemize}

\item Cross reactivity:
\begin{itemize}
    \item Gaevert et al, Viruses 2021: stochastic model of cross-reactivity between TCRs and pMHCs, showing that different mechanisms of biased or unbiased cross-reactivity can generate immunodominance hierarchies as seen in naturalistic infection
\end{itemize}

\end{enumerate}

\section*{Additional notes for introduction}

Here are some additional facts to synthesize for the introduction:

\begin{itemize}
    \item Low affinity T cells escape tolerance and cause autoimmunity (Zehn and Bevan, Immunity 2015)
    \item "No mathematical model of biology is completely accurate, but some can be still useful" - the use of mathematical and computational models can help provide theoretical frameworks that can, together with experimental investigation, help piece together the underlying workings of the system under study.
    \item Discussion? As a computational immunologist, designing models often requires. Benefits from a "ground-up" approach, where we start with the simplest possible model and gradually increase the complexity until the model can reproduce key experimental data, and generate testable predictions.
    \item Discussion? There's a balance in model design, where it's often non-trivial to know, when trying to keep a model simple without overloading the number of parameters and having it become intractable, what aspects of the biology are essential components that should be explicitly included, vs. which ones, while undoubtedly important to the system under study, can be safely omitted without drastic implications on the conclusions of the model. Of course, with nonlinear systems, it is often impossible to predict this without investigating it, and so no model is complete/there is always room for more complexity.
    \item Modelling is a two-way communication with experimentalists. The work undertaken in all 3 projects involved back and forth with experiments and modelling
    \item For Ch 4: Roy Anderson and Robert May SIR model pioneering late 1970s/1980s.
\end{itemize}
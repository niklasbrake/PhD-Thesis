\chapter{General discussion}
\label{sec:conclusion}

\section{Summary of research findings}

Despite the many advancements made to date in understanding how T cell antigen specificity and heterogeneity shape their dynamic anti-microbial or immunosuppressive responses, much remains that is yet to be uncovered. Across the three studies presented in this thesis, our primary aim was to use computational modelling to untangle some of these complexities, thus informing their resulting function in infection and autoimmunity and providing incremental advancements toward our general understanding of these phenomena.

First, in Chapter~\ref{sec:Tr1}, we sought to understand what biological factors could explain pMHC-NP treatment outcomes upon induction of autoimmune disorders in mice. We developed a compartmental population model to ask how Tr1 allocation and Tr1-mediated immunosuppression might together explain those experimental observations upon pMHC-NP administration. We found that Tr1 antigen specificity and binding avidity to APCs resulting from differences in cognate pMHC expression levels, together with recruitment kinetics into sites of autoimmune destruction, could account for differences observed in NP-therapy success vs. failure across experimental conditions.

Next, in Chapter~\ref{sec:AvC}, we were interested in adapting our model to study the effects of pMHC binding strengths in shaping T cell responses to acute or chronic pathogen replication. Together with experimental support, our model revealed differential evolution in pMHC reactivity profiles depending on infection duration, wherein chronic infection skewed the response toward lower-affinity T cells. Additionally, we showed that chronic pathogen clearance was delayed in mice whose T cells lacked TdT-dependent N-diversification, consistent with our \textit{in silico} predictions of TCR repertoires lacking T cells with lower pMHC reactivity; these results thus support the existence of a previously undescribed benefit for TdT-dependent TCR repertoire diversification.

Finally, in Chapter~\ref{sec:VE}, we wanted to computationally investigate whether TdT-mediated TCR repertoire diversification might provide additional benefits for pathogen control, namely by filling in repertoire ``holes'' and preventing mutating pathogens from escaping the T cell response. We found that, under certain conditions, fewer unique TCR clonotypes can indeed make a host more susceptible to emerging escape variants, and that this effect ultimately depends upon a balance between pathogen-intrinsic dynamics and T cell-mediated control and selective pressure.

While our research was able to provide new insights into how T cell antigen specificity and heterogeneity shapes inflammatory or immunoregulatory responses, there remain several unanswered questions worth investigating. In the following sections, the broader implications of our work will be discussed, with an emphasis on how future studies could explore these ideas to further advance our understanding of the behavior and function of T cell dynamics, and beyond.


\section{Looking forward -- avenues for future studies}

\subsection{Tr1-mediated immunosuppression -- implications for therapeutic potential}

In Chapter~\ref{sec:Tr1}, the pMHC-NP based treatment protocols whose outcomes we aimed to reproduce with our model involved simultaneous induction of multiple autoimmune responses within the same mouse. While deliberately triggering the rapid onset of two autoimmune disorders simultaneously may not directly mimic the development of autoimmunity in real biological systems, these results nonetheless have real implications for the clinical translational potential of pMHC-NP therapy. Recall that, in our model, we described one possible outcome that we termed ``competitive autoimmunity'', wherein scarce Tr1 resources can lead to ubiquitous failure for pMHC-NPs to resolve inflammation in multi-autoimmunity despite possibly succeeding in the case of single autoimmune disorders at a time. We argued that the failure of MOG-NPs to treat CNS autoimmunity in the presence of severe liver inflammation, in spite of efficient immunoregulation arising from these MOG-NPs when the CNS alone is impacted, is a direct result of competitive autoimmunity; here, liver inflammation can non-specifically sequester Tr1s and prevent them from reaching the brain (and its draining lymph nodes), thus rendering them unable to exert their immunosuppressive effects in the latter.

Importantly, competitive autoimmunity and non-specific sequestering of Tr1s may complicate efforts to treat autoimmune disorders in humans using pMHC-based NPs, perhaps most obviously in the case of systemic autoimmune disorders that affect multiple organs of the body. A prominent example of this would be systemic lupus erythematosus, a multi-organ autoimmune disorder targeting connective tissue (present throughout the entire body) with an estimated 3.17~million people affected worldwide~\cite{fatoye2022global,tian2023global}. Here, auto-antigen specific Tr1s would need to migrate to multiple sites throughout the body, and in sufficient numbers, to effectively halt the progression of autoimmune destruction. If the number of antigen-experienced Tr1 precursors targeted by pMHC-NP is insufficient to meet the demand, then our model predicts that these Tr1 cells could ubiquitously fail to produce a noticeable immunoregulatory effect. As such, additional studies are needed to determine the outcomes of pMHC-NP treatment in mouse models of systemic autoimmunity. One potential way to achieve larger numbers of Tr1 cells from antigen-experienced precursors might involve using a mixture of pMHC-NPs containing multiple self-pMHC epitopes, and thus theoretically reprogramming Tr1s from a larger number of antigen-experienced autoreactive T cells of broader specificity; however, whether such a strategy could overcome competitive autoimmunity remains to be seen. 

Based on our results in Chapter~\ref{sec:Tr1}, another major obstacle that may impeded upon the therapeutic potential of pMHC-NPs in humans involves the importance of timing in determining the success of Tr1-mediated immunosuppression. Importantly, we showed that transient delays in auto-antigen specific Tr1 recruitment can lead to treatment failure and thus persistent autoimmunity, as observed in the CNS in the case of PDC-NP treatment in mice with simultaneous liver autoimmunity~\cite{umeshappa2020ubiquitous}. Subsequent experiments showed that sequential, as opposed to simultaneous, induction of liver and CNS autoimmunity curtails treatment failure in the CNS, corroborating our model prediction and highlighting the importance of the \textit{kinetics} of Tr1 allocation. This is of particular relevance, since autoimmune diagnoses may lag disease onset by several years, by which time substantial tissue damage may have already been incurred~\cite{rose2016prediction}. Thus, for practical clinical usage, more work is needed to assess and to improve the ability for pMHC-NP-induced Tr1s to treat autoimmunity long after disease onset. 

In a recent follow-up study conducted by the Santamaria laboratory, Solé et al. showed that, while Tr1 antigen specificity is indeed constrained by the pMHC displayed on the NPs as previously inferred, the induced Tr1 population was oligoclonal (meaning that it was composed of multiple unique TCR clonotypes) as revealed by single cell RNA sequencing~\cite{sole2023transcriptional}. Of note, in Chapter~\ref{sec:Tr1} our model grouped together all Tr1 clones by considering a population average, and did not examine the possible effects of heterogeneity within the Tr1 population. However, similar to our approach in Chapters~\ref{sec:AvC} and~\ref{sec:VE}, future studies could explicitly distinguish between different Tr1 clonotypes in order to study their unique contributions in the context of immunoregulation. For example, while Tr1-mediated autoimmune suppression relies on cognate pMHC expression on APCs, its modes of suppression are extensive and additionally involve local regulatory B cell formation. Thus, similar to how differences in pMHC affinity can influence helper T cell lineage and function~\cite{martin2013highly,van2014t,sood2019differential,rogers2021pre,snook2018tcr,ditoro2018differential,van2016tcr}, perhaps such heterogeneity present within Tr1 populations may analogously give rise to distinct contributions to the overall response.

While pMHC-NP therapy constitutes a specific system within which we could study Tr1-mediated immunosuppression, it should be noted that auto-antigen specific Tr1s can be found in circulation in both mice and in humans at homeostasis~\cite{gagliani2013coexpression,freeborn2022type}. Thus, investigating their dynamics in the case of pMHC-NP treatment provides an opportunity to better understand the immunoregulatory role that these cells carry out more generally. Specifically, our modeling work corroborated the importance of Tr1 antigen specificity in dictating their regulatory potential locally, with Tr1 avidity (through a combination of TCR affinity and cognate antigen expression levels) playing a key role in its suppressive potential. How Tr1s might differentially contribute to the maintenance of peripheral tolerance at homeostasis, when compared to the better-studied FOXP3\pos{} Tregs, remains poorly described and is worth further theoretical and experimental investigation.

\subsection{Low-affinity T cells -- beyond chronic infection control}

In Chapter~\ref{sec:AvC}, we showed that T cells with low foreign antigen reactivity, being less prone to exhaustion than higher-affinity T cells, preferentially contribute to the control of chronic replicating pathogens. We argued that TdT-mediated N-nucleotide diversity generates TCRs with lower pMHC binding strengths on average, and that these therefore benefit the host immune responses in chronic infection. However, it should be noted that chronic antigen stimulation is not only present specifically in the context of chronic \textit{infection}, with two important counterexamples being autoimmune disorders and cancer. The question then becomes: might TdT-dependent T cells be making important contributions in such settings as well, relative to their TdT-independent counterparts? While it remains to be explicitly demonstrated, there is evidence to suggest that this could indeed be the case.

Recall that central tolerance preferentially purges higher-affinity T cells via negative selection and nTreg induction, while peripheral tolerance mechanisms such as ignorance and anergy (Section~\ref{sec:intro_autoimmunity_peripheralTolerance}) are needed to keep lower-affinity autoreactive T cells in check~\cite{zehn2006t,leube2023single}. Importantly, under the right conditions (e.g., breakdown of ignorance in the presence of high auto-antigen levels), these autoreactive T cells can also escape peripheral tolerance and cause autoimmunity~\cite{zehn2006t}. Hence, one prediction might be that autoreactive T cells in TdT-deficient mice, having higher affinity for self-pMHC on average, would be more efficiently deleted at the level of the thymus, with fewer circulating autoreactive T cells in the periphery. Consistent with this hypothesis, a number of studies using different lupus-prone or diabetes-prone mouse models have shown that autoimmune incidence and/or severity was consistently lower in these mice when TdT was absent~\cite{conde1998terminal,feeney2001terminal,robey2004terminal}. In using TdT to diversify the host TCR repertoire, there may therefore be an evolutionary trade-off between its beneficial role(s) in chronic pathogen control vs. increased susceptibility to autoimmunity.

Low-affinity T cells may also be important for immune responses to cancer. In this context, T cells recognize transformed tumour cells via their expression of tumour-associated antigens (TAAs) that, in most cases, are simply over-expressed self-antigens; thus these T cells are subjected to central tolerance mechanisms in the thymus against these TAAs~\cite{mcmahan2007mobilizing,martinez2018pd}. Therefore, as with autoimmune responses, responding T cells that do escape negative selection will bind TAAs with lower affinity when compared to anti-microbial T cells specific for foreign pMHC~\cite{mcmahan2007mobilizing}. While anti-tumour immunity is complex and cancer cells have the ability to suppress T cell responses through a large variety of mechanisms (reviewed in~\cite{kim2022evasion}), there may be important parallels our findings with regard to low-affinity T cells in chronic pathogen control. For example, a number of groups have demonstrated differential induction of T cell exhaustion based on TCR affinity for pMHC on tumour cells, wherein higher TCR signal strengths lead to more terminally exhausted phenotypes~\cite{martinez2018pd,shakiba2021tcr,hay2023low}. In one such study, Shakiba et al. argue that there is a ``sweet-spot'' in the magnitude of TCR affinity wherein sufficient effector potential, together with reduced susceptibility to exhaustion, provide optimal control of tumour growth~\cite{shakiba2021tcr}. These findings, in conjunction with our own results highlighted in Chapter~\ref{sec:AvC}, may therefore have implications in the optimal design of engineered TCRs for cancer immunotherapy.


\subsection{Uncovering the many faces of TdT?}

Our computational model in Chapter~\ref{sec:AvC}, together with indirect experimental evidence also presented in Chapter~\ref{sec:AvC} as well as by other groups (see Section~\ref{sec:AvC_discussion}), suggest that TdT-dependent T cells are of lower pMHC reactivity on average. However, direct experimental demonstration of differences in TCR affinity between the two groups using available techniques has been challenging. For example, staining of WT vs. TdT KO T cells with LCMV glycoprotein (GP)-derived pMHC tetramers did not produce observable differences in binding intensities for either CD4\pos{} or CD8\pos{} T cells (data not shown), contrary to what might be expected if TdT KO T cells have higher-affinity TCRs. That said, there are several reasons why pMHC tetramers may not be suitable for assessing these differences; these were outlined in Section~\ref{sec:intro_affinity_TcellAvidity} but briefly, tetramers introduce biases by capturing only those T cells possessing higher-affinity TCRs, and with limited pMHC epitope specificity (as opposed the entire responding T cell population spanning multiple epitope specificities within the full spectrum of binding affinities). More direct and more sensitive measurements of TCR affinities techniques, e.g., via adhesion frequency assays across multiple epitope specificities, might be useful tools for establishing a clearer link between N-additions and antigen binding strengths as predicted by our model (although these methods are also not without their limitations, as discussed in Section~\ref{sec:intro_affinity}).

It should also be restated that TdT-independent TCRs are more cross-reactive, binding to a significantly larger number of antigens within a large pMHC library compared to N-diversified TCRs~\cite{gavin1995increased}. Higher cross-reactivity can be thought of as compensating for the reduced TCR diversity to similarly maintain adequate antigen recognition potential in TCRs formed in the absence of TdT (as is the case, for example, in neonatal TCR repertoires). However, the link between TCR affinity for foreign pMHC and its cross-reactivity is not entirely clear. In one modelling study by Xu et al., the authors argue using an amino-acid string model (a more realistic adaptation of the digit-string model presented in Section~\ref{sec:intro_affinity_selfVsForeign}) that the average binding energies of highly cross-reactive TCRs to a library of peptides, given their amino acid compositions, were greater than less cross reactive TCRs~\cite{xu2019broad}; this would be consistent with our prediction that TdT-independent TCRs, which happen to be more cross-reactive, are also of higher affinity on average. However, establishing this link experimentally is no simple undertaking, with one issue being that even large peptide libraries represent minute fractions of the possible pMHC epitopes that a T cell may encounter. This is further complicated by the fact that a given TCR being ``low-affinity'' for one pMHC may have higher affinity for a different one and vice-versa~\cite{martinez2015lower}.

Though we argued that TdT benefits host responses to chronic infection through the inclusion of lower-affinity T cells into the TCR repertoire, we acknowledged that this is likely not the sole evolutionary advantage of TCR repertoire diversification by TdT. Subsequently, in Chapter~\ref{sec:VE}, we proposed with our computational model that TdT-sufficient TCR repertoires may be better poised to recognize and respond to mutated pMHC epitopes and thus prevent pathogen immune escape. To date, this hypothesis has yet to be proven experimentally; at the time of writing, experiments are underway to establish whether TdT KO mice infected with LCMV-Cl13, a chronic RNA virus known to exhibit antigenic variation at pMHC epitope regions of its genome \textit{in vivo}~\cite{smyth2021characterization}, are more susceptible to the emergence of these escape mutants than are their TdT-sufficient counterparts as the model might predict. Of note, the concept of T cells responses in the presence of antigenic variation via epitope mutation is relevant beyond the control of RNA viruses. For example, tumours may also undergo genetic mutation in order to evade T cell recognition~\cite{dunn2002cancer,finn2018believer,hirschhorn2023t}. Therefore, it would additionally be interesting to study whether TdT-deficient TCR repertoires might also exacerbate immune escape in various cancer settings.

Undoubtedly, there remains much to uncover with regard to the functional role of TdT-mediated lymphocyte receptor repertoire diversification. For example, recall that TdT is equally involved in facilitating N-nucleotide additions during V(D)J recombination of the B cell receptor (BCR) repertoire, yet, as with T cell responses, the ability for B cells to produce pathogen-specific neutralizing antibodies in response to acute pathogens is not noticeably affected in TdT KO mice~\cite{gilfillan1995efficient}. Surprisingly, when full TdT KO mice were infected with chronic LCMV-Cl13 (as opposed to chimeric mice where TdT deficiency was restricted to T cells as in Chapter~\ref{sec:AvC}), no difference was observed in the viral titres relative to WT controls (data not shown). This raises the question as to whether there are fundamental differences in how T and B cell responses are modulated by TdT. One hypothesis is that, while the average affinity of antibodies produced by TdT KO B cells could also be higher, this might have the opposite effect on chronic pathogen control, therefore compensating for delayed clearance by TdT-deficient T cells. In fact, B cell receptors (the membrane-bound form of antibodies) undergo affinity maturation in a process known as ``somatic hypermutation'' that purposefully leads to higher affinity binding of antibodies to pathogen-derived antigens~\cite{wagner1996somatic,meffre2001somatic,di2007molecular}. Perhaps, then, if BCRs in TdT KO mice are of higher affinity to start, the process of somatic hypermutation leads to the production of even higher affinity antibodies, which in turn results in improved viral neutralizing capacity during chronic infection. Restriction of TdT KO to B cells, for example in mixed bone marrow chimeras, will be needed to assess whether the reverse effect is seen from T cells, i.e., whether TdT-deficient BCR repertoires lead to improved chronic pathogen control.

\subsection{From within- to between-hosts -- epidemiological impact of TCR diversity?}

In Chapter~\ref{sec:VE}, we focused on the effects of TCR repertoire diversity on the evolution of rapidly mutating pathogens within a single infected host. Having shown with our model that reduced repertoire diversity may in fact lead to higher emergence rates of dominant mutants, and that these mutants bypass T cell recognition to a larger extent, intriguing questions arise. First, does TdT-deficiency, and hence lowered TCR repertoire diversity, increase the likelihood of pathogen immune escape within an infected host, as might be suggested by the model? And second, could this have any implications for pathogen evolution and spread across hosts within a susceptible population? In other words, might the evolutionary advantage of TdT-mediated TCR repertoire diversification additionally extend beyond the context of within-host T cell responses, helping to protect an entire population by conferring additional protective immunity at an epidemiological scale?

Importantly, linking within-host adaptation to between-host evolution is no straightforward task. The ability for a given pathogen to bypass local immune pressure by generating mutant strains does not necessarily translate to increased transmissibility of those strains from one host to the next. For example, in the case of HIV, the original infecting strain of the virus (or variants otherwise generated early in infection) are more likely to be transmitted to subsequent hosts than are late-stage viral variants~\cite{redd2012previously,theys2018impact}. Furthermore, while imbalanced (ladder-like) phylodynamic trees within hosts reveal dramatic selection pressures shaping HIV evolution, between-host phylodynamic trees do not follow this pattern, instead possessing a more star-like, balanced topology~\cite{theys2018impact,grenfell2004unifying}; these contrasting data thus reveal distinct patterns of within- and between-host evolution. Such differences arise from the fact that pathogen spread and evolution at the population level will be dictated not only by individual host immunity, but also by other important dynamics including transmission bottlenecks, contact rates between carriers and non-carriers, community susceptibility (e.g., the degree of ``herd immunity''), and population-level migration patterns across geographical scales~\cite{grenfell2004unifying,saad2022immuno,bergstrom1999transmission,cardenas2022genomic}.

While it is clear that epidemiological factors determining the progression of infectious diseases are highly complex and span many biological and ecological scales, within-host dynamics (and, hence, T cell dynamics) undoubtedly play an important role in shaping disease epidemiology. With regard to TCR repertoire diversity, there is reason to believe that TdT-deficiency in a susceptible population might render it more vulnerable to mutating pathogens than an otherwise equivalent, TdT-sufficient population. This is because TdT-dependent TCRs with more N-additions that are found in one individual are less likely to be present in others, even in genetically identical organisms~\cite{yassai2000thymocyte,yassai2002molecular,quigley2010convergent} -- these are termed ``private'' TCRs (in contrast to ``public'' TCRs that are shared across multiple individual TCR repertoires). Thus, mutations that escape recognition by private TCRs in one host might still illicit a sufficient T cell response in another, given that their TCR repertoires do not fully overlap. This is consistent with the observation that public TCRs facilitate immune escape by simian immunodeficiency virus (SIV, the HIV analogue in non-human primates)~\cite{price2004t,li2012determinants}. To experimentally test whether a population possessing more-public TCR repertoires are, as a community, more susceptible to mutating pathogens than are TdT-sufficient populations, serial passaging experiments can be employed; here, a virus such as LCMV-Cl13 could be extracted from an infected host, purified, and transferred into a new susceptible host to mimic a chain of transmission, with this process repeated over multiple generations and viral diversity sequenced at each iteration. In doing so, key metrics such as mutant emergence rates, pathogen virulence/severity, viral loads, and responding T cell phenotypes can be assessed. Of note, comparing the outcomes of such serial passaging experiments between WT and TdT KO mice might reveal whether there exist important population-level differences in the response to a mutating pathogen; if this is the case, this would reveal another novel, and important, evolutionary advantage for TCR repertoire diversification via TdT.


\section{Concluding remarks}

The notions of T cell antigen specificity, heterogeneity, pMHC reactivity, and immune escape are all very complex, fascinating, and ongoing areas of research. This thesis, and the many years of study that it represents, outlines what we have done to better understand certain aspects of these phenomena, using computational tools and techniques to support and build upon experimental knowledge. As with any scientific methodology, computational modelling possesses strengths and weaknesses alike (both of which I have become well acquainted with throughout my degree). The models presented herein represent the culmination of my efforts throughout the years to untangle these intriguing, and puzzling, features of T cell biology as they pertain to infection and autoimmunity. I must highlight that no model is perfect, and the ones I have developed are certainly no exception to this rule; that said, there could never be hope to perfectly encapsulate all aspects of T cell responses in all their nuances, barring none, as such a feat would be truly impossible. Rather, the aim of these models was to provide mechanistic insights into select properties and functions of T cells, offering some clarity while highlighting remaining gaps in our knowledge. Though our work has helped to shed some light on how antigen specificity shapes T cell responses in the contexts outlined throughout the thesis, it goes without saying that there is still much work to be done before we, as a field, have fully grasp the complexities and intricacies that exist within the wonderful world of T cell dynamics.
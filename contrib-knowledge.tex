\chapter*{Contribution to original knowledge}
\addcontentsline{toc}{chapter}{Contribution to original knowledge}
\fancyhead[L]{{\color{gray}\nouppercase{Contribution to original knowledge}}}

\hypersetup{linkcolor=seccolor}

The work presented in this thesis provides novel contributions to the field of T cell immunology, using computational modelling to further our understanding of the roles of T cell antigen specificity and heterogeneity in controlling infections and preventing autoimmune disorders.

In Chapter~\ref{sec:Tr1}, we investigated the dynamics of regulatory T cell-mediated autoimmune suppression in multiple simultaneous autoimmune disorders. Prior to this work, it was shown that a class of regulatory T cells, called T-regulatory type 1 cells (Tr1s), could be generated \textit{in vivo} by administering disease-relevant nanoparticles into autoimmune mice, and that these Tr1s could comprehensively suppress tissue-restricted autoimmunity while sparing global immune function. However, it was not known why treatment of mice with multiple simultaneous autoimmune disorders failed in certain cases, even when Tr1-mediated immunoregulation successfully reverses those same autoimmune disorders one at a time. In this work, we found that Tr1 antigen specificity, together with Tr1 recruitment kinetics and antigen shedding levels within autoimmune tissues, could predict treatment outcomes and explain the discrepancies observed as a function of the number of autoimmune organs.

In Chapter~\ref{sec:AvC}, we sought to investigate how T cells of different antigen binding strengths might variably contribute to the control of acute or chronic pathogen replication. Specifically, it was not known whether T cell receptor repertoire diversification by terminal deoxynucleotidyl transferase (TdT), whose evolutionary benefit had remained elusive, introduces biases in foreign antigen binding strengths or whether such biases might specifically benefit chronic infection control without altering the T cell response to acute pathogens. In this work, we showed that T cells with relatively weaker foreign antigen binding strength preferentially contribute to the control of chronic infections, and that this can result from reduced propensity for these cells to become functionally exhausted during chronic antigen stimulation. We further showed that TdT deficiency indeed delays chronic pathogen clearance while acute control was unaffected, supporting the notion that TdT-dependent T cell receptors bind their cognate ligands with lower affinity on average.

In Chapter~\ref{sec:VE}, we probed an additional hypothesis for the benefit of T cell receptor repertoire diversification by TdT, namely whether this added diversity might plug ``holes’’ in the repertoire that mutating pathogens would otherwise exploit. Prior to this study, the manner in which changes to the diversity of a host’s T cell receptor repertoire influences the rate of pathogen immune escape remained poorly understood. In this work, we predicted that intermediate numbers of unique T cell clones indeed have the potential to optimally promote the emergence of escape mutants relative to both lower and higher repertoire diversities. Using our model, we proposed that the key determinants driving pathogen immune escape involve a combination of pathogen-intrinsic dynamics (such as replication and mutation rates) together with T cell-mediated selection pressures, the latter being driven by the number of unique T cell clones responding to the infection.

\hypersetup{linkcolor=black}
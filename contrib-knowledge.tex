\chapter*{Contribution to original knowledge}
\addcontentsline{toc}{chapter}{Contribution to original knowledge}
\fancyhead[L]{{\color{gray}\nouppercase{Contribution to original knowledge}}}

The neural basis of EEG is still incompletely understood. Of particular recent interest, EEG signals exhibit an overall spectral background trend, whose origin and interpretation is intensely debated. The major theories on the EEG spectral trend rely principally on conceptual and phenomenological modelling. Broadly speaking, this thesis contributes to our understanding of the neural basis of EEG by providing a comprehensive biophysical theory of how the EEG spectral trend can be generated by neural activity. This theory is built from several novel findings:

Firstly, scientist had argued conceptually whether EEG signals can reflect asynchronous background neural activity. This thesis presents biophysical calculations that show it is impossible for asynchronous neural activity to be detected on an EEG, providing the first argument against this interpretation that combines both biophysical EEG simulations and experimentally determined physiological parameters of neural activity (\autoref{sec:natcomms}). 

Secondly, it was not known whether EEG signals can reflect non-oscillatory neural activity. This thesis presents the results of numerical simulations which demonstrate that aperiodic activity of a similar nature to that observed experimentally in animals can produce detectable EEG signals, providing the first evidence that EEG signals can theoretically reflect aperiodic neural activity (\autoref{sec:natcomms}).

Thirdly, it is debated how the EEG spectral trend should be corrected for when quantifying brain rhythms. This thesis examines the spectral changes known to be caused by the administration of the drug propofol, a general anesthetic. The results illustrate how broadband changes in EEG spectra arise from propofol's known pharmacology using biophysical models of electric field generation. These simulations demonstrate how the changes induced by propofol at the single cell level alter the production of rhythmic EEG signals, and provide the first concrete, biophysically justified case where changes in the EEG spectra trend corrupt brain rhythms estimates (\autoref{sec:natcomms}).

Fourthly, the thesis proposes how the effects of propofol on the EEG spectral trend should be rectified. Quantitatively accounting for these changes on EEG collected from patients undergoing propofol anesthesia revealed for the first time that loss of consciousness from propofol is time locked exclusively to delta rhythms, in contrast to past studies that have implicated both alpha and delta rhythms  (\autoref{sec:natcomms}).

Finally, it was not known whether or not action potentials can generate detectable EEG signals. Dissenting opinions were based on conceptual arguments, while concurring opinions were based on extrapolation from local field recordings. This thesis details the results of biophysical simulations and EEG forward modelling, which demonstrate that action potentials cannot generate aperiodic EEG signals (\autoref{sec:apEEG}). These simulations also, for the first time, define a frequency range where EEG rhythms can reasonably be generated by action potentials (\autoref{sec:apEEG}).

In addition to the main chapters of the thesis, I also include an appendix which details work done during my PhD that is not directly related to the neural basis of EEG. This work produced original knowledge about the molecular basis of voltage sensing in proteins called voltage-gated sodium channel. This work showed for the first time that the molecular domains of the sodium channel protein have different functions depending on the protein isoform. Specifically, most investigations had been performed on the sodium channel expressed by skeletal muscle, which had concluded that the fourth voltage sensing domain is alone responsible for channel inactivation. The results presented in \autoref{sec:Nav} illustrate that in the cardiac sodium channel, both the third and fourth domain are necessary for channel inactivation. Additionally, the work details how closed-state inactivation is variable among sodium channel isoforms. Experimental evidence from the skeletal muscle channel indicated that the fourth voltage sensing domain moves extremely slowly in response to changes in the electric field relative to the other three domains, which would prevent closed state inactivation. We show that the cardiac sodium channel has a very high propensity for closed state inactivation, indicating that the dynamics of the various voltage sensing domains must differ among channel isoforms (\autoref{sec:Nav}).


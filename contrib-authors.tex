\chapter*{Contribution of authors}
\addcontentsline{toc}{chapter}{Contribution of authors}
\fancyhead[L]{{\color{gray}\nouppercase{Contribution of authors}}}

\setlength{\parindent}{0pt}
\setlength{\parskip}{3pt}

This thesis is presented as a manuscript-based thesis. Below, I describe individual contributions to each manuscript that together constitute the main chapters of my thesis. The final manuscript is included as an appendix as it does not directly relate to the main focus of my thesis.

\vspace{.5em} 
Chapter 2 \hrule

\noindent
\hangindent=1cm
Brake N, Duc F, Rokos A, Arseneau F, Shahiri S, Khadra A, and Plourde G. A neurophysiological basis for aperiodic EEG and the background spectral trend. \textit{Nature Communications} \textbf{15}, 1514 (2024). \url{https://doi.org/10.1038/s41467-024-45922-8}

{\small \noindent \textbf{Brake N}: Conceptualization, Data curation, Formal analysis, Investigation, Methodology, Software, Visualization, Writing—original draft, Writing—review and editing. \textbf{Duc F}: Investigation. \textbf{Rokos A}: Investigation. \textbf{Areseau F}: Investigation. \textbf{Shahiri S}: Investigation. \textbf{Khadra A}: Conceptualization, Funding acquisition, Project administration, Supervision, Writing—review and editing. \textbf{Plourde G}: Conceptualization, Data curation, Funding acquisition, Investigation, Project administration, Supervision, Writing—review and editing.}

\vspace{.5em} Chapter 3 \hrule

\noindent
\hangindent=1cm
Brake N and Khadra A. Contributions of action potentials to scalp EEG: theory and biophysical simulations. \textit{eLife}. Available from: \url{https://doi.org/10.1101/2024.05.28.596262}

{\small \textbf{Brake N}: Conceptualization, Methodology, Investigation, Writing- Original draft preparation, Visualization. \textbf{Khadra A}: Supervision, Funding Acquisition, Writing- Reviewing and Editing.}

\vspace{.5em} 

Appendix A \hrule

\noindent
\hangindent=1cm
Brake N*, Mancino AS*, Yan Y, Shimomura T, Kubo Y, Khadra A, and Bowie D. Closed-state inactivation of cardiac, skeletal, and neuronal sodium channels is isoform specific. \textit{Journal of General Physiology} \textbf{154}, 7: e202112921 (2022). \url{https://doi.org/10.1085/jgp.202112921}

{\small \textbf{Brake N}: Conceptualization, Methodology, Formal Analysis, Investigation, Writing—Original Draft, Visualization.  \textbf{Mancino AS}: Conceptualization, Formal Analysis, Investigation. \textbf{Yan Y}: Formal analysis, Investigation. \textbf{Shimomura T}: Methodology, Formal analysis. \textbf{Kubo Y}: Methodology, Funding acquisition. \textbf{Khadra A}: Funding Acquisition, Supervision. \textbf{Bowie D}: Conceptualization, Funding Acquisition, Writing—Original Draft, Supervision, Project Administration.}
